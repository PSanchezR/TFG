\chapter*{}
%\thispagestyle{empty}
%\cleardoublepage

%\thispagestyle{empty}

%\input{portada/portada_2}



%\cleardoublepage
\thispagestyle{empty}

\begin{center}
{\large\bfseries  MEDA-Toolbox: Interfaz Gráfica en Matlab para la detección de anomalías en red}\\
\end{center}
\begin{center}
Pablo Sánchez Robles  (alumno)\\
\end{center}

%\vspace{0.7cm}
\noindent{\textbf{Palabras clave}: Análisis exploratorio de datos, MEDA-Tooblox, Big Data, detección de anomalías en red, datos de seguridad en red, AED, GUIDE}

\vspace{0.7cm}
\noindent{\textbf{Resumen}}\\

En el presente documento se pretende desarrollar una interfaz gráfica para análisis de BIG DATA haciendo uso de la Toolbox MEDA, y orientar dicha interfaz a la detección de anomalías en redes de computadores con un amplio número de conexiones y usuarios finales.
\bigskip

Se comenzará por estudiar las metodologías que se utilizarán para la disminución del tamaño del problema y para la identificación de los elementos notables, ya sea porque generan las anomalías en el funcionamiento correcto del sistema o porque no se tienen encuadrados dentro de los parámetros normales. Se dará una introducción sobre el concepto de BIG DATA y los grandes conjuntos de datos.
\bigskip

También se hará un breve resumen de la historia de la seguridad informática así como de los elementos a tener en cuenta dentro de los sistemas para a detección de las fallas en la seguridad.
\bigskip

Se hará la ingeniería del software pertinente para el desarrollo de la interfaz y se finalizará con un caso de estudio que pondrá a prueba lo desarrollado a lo largo de todo el trabajo.

\cleardoublepage


\thispagestyle{empty}


\begin{center}
{\large\bfseries MEDA-Toolbox: Graphic user interface in Matlab for network anomaly detection}\\
\end{center}
\begin{center}
Pablo Sánchez Robles (student)\\
\end{center}

%\vspace{0.7cm}
\noindent{\textbf{Keywords}: Exploratory Data Analisys, MEDA-Toolbox, Big Data, Network Anomaly Detection, Security data network , EDA, GUIDE}\\

\vspace{0.7cm}
\noindent{\textbf{Abstract}}\\

Herein it is intended to develop a graphical interface for BIG DATA analysis by using the MEDA toolbox and head such an  interface for the detection of anomalies in computer networks with a large number of connections and end users.
\bigskip

We will start studying the methodologies to be used for reducing the size of the problem and for identifying the significant items,either because they generate anomalies in the proper functioning of the system or because they are not framed within  normal parameters . An  introduction to the concept of BIG DATA and large data sets will be given.
\bigskip

A brief summary of the history of the computer security as well as the elements to be considered within the systems for the detection of security breaches will also be done.
\bigskip

Engineering to the development of   relevant interface software will be made and  by ending up with a case study that  will test everything  that has been  developoled throughout the work
\bigskip

\chapter*{}
\thispagestyle{empty}

\noindent\rule[-1ex]{\textwidth}{2pt}\\[4.5ex]

Yo, \textbf{Pablo Sánchez Robles}, alumno de la titulación Ingenría Informática de la \textbf{Escuela Técnica Superior
de Ingenierías Informática y de Telecomunicación de la Universidad de Granada}, con DNI 75159703-G , autorizo la
ubicación de la siguiente copia de mi Trabajo Fin de Grado en la biblioteca del centro para que pueda ser
consultada por las personas que lo deseen.

\vspace{6cm}

\noindent Fdo: Pablo Sánchez Robles

\vspace{2cm}

\begin{flushright}
Granada a X de mes de 2015 .
\end{flushright}


\chapter*{}
\thispagestyle{empty}

\noindent\rule[-1ex]{\textwidth}{2pt}\\[4.5ex]

D. \textbf{José Camacho Páez}, Profesor del Área de Ingeniería Telemática del Departamento Teoría de la Señal, Telemática y Comunicaciones de la Universidad de Granada.

\vspace{0.5cm}

\vspace{0.5cm}

\textbf{Informan:}

\vspace{0.5cm}

Que el presente trabajo, titulado \textit{\textbf{MEDA-Toolbox: Interfaz Gráfica en Matlab para la detección de anomalías en red}},
ha sido realizado bajo su supervisión por \textbf{Pablo Sánchez Robles}, y autorizamos la defensa de dicho trabajo ante el tribunal
que corresponda.

\vspace{0.5cm}

Y para que conste, expiden y firman el presente informe en Granada a X de mes de 2015 .

\vspace{1cm}

\textbf{Los directores:}

\vspace{5cm}

\noindent \textbf{José Camacho Páez}

\chapter*{Agradecimientos}
\thispagestyle{empty}

       \vspace{1cm}


	A mis padres, mi hermana y mi cuñado por aguantarme día a día como lo han hecho y por estar siempre ahí para todo lo que me ha hecho falta.
	\bigskip
	
	A mis amigos por ser la vía de escape en los momentos difíciles y no fallarme en ningún momento.
	\bigskip
	
	A mis compañeros de la ETSIIT por darlo todo cuando se les necesita y por hacer del trabajo en equipo un sistema de vida.
	\bigskip
	
	A mis abuelos por inculcarme el valor por el estudio desde que tengo uso de razón.
	\bigskip
	
	Especialmente a todos los que ya no están porque la marca que han dejado jamás se perderá.
 

